\chapter{Implementazione}
\label{chap:implementazione}
\begin{minipage}{12cm}\textit{Sono di seguito presentati dettagli riguardanti l'implementazione dei passi principali eseguiti dal programma in esame. }
\end{minipage}

\section{Avvio iniziale}
All'avvio dell'applicazione, viene visualizzata la schermata principale di configurazione; vengono inoltre raccolte e scritte sul file \textit{sys\_info.txt} informazioni riguardanti il sistema operativo della macchina e le versioni dei browser installati.

\vspace{0.5cm}
\begin{lstlisting}[caption=Funzione main,captionpos=b]
public class Byob_v1 {

	public static void main(String[] args) {
		/**Start the GUI*/
		GUI frame = new GUI();
		final Toolkit toolkit = Toolkit.getDefaultToolkit();
		final Dimension screenSize = toolkit.getScreenSize();
		int x = (screenSize.width - frame.getWidth()) / 2;
		int y = (screenSize.height - frame.getHeight()) / 2;
		frame.setTitle("BYOB v_1");
		frame.setLocation(x, y);
		frame.setVisible(true);
		/**Gather system informations and write them on sys_info.txt*/
		Tools.writeInfoFile("sys_info.txt");
	}

}
\end{lstlisting}

\section{Informazioni di sistema}

Il sistema operativo in esecuzione sulla macchina \`{e} restituito dalla funzione \textit{Tools.getOs()}.

\vspace{0.5cm}
\begin{lstlisting}
    public static String getOs(){
	    return System.getProperty("os.name");
    }
\end{lstlisting}

Per riuscire ad identificare i browser installati, sono state adottate strategie differenti per ogni sistema operativo individuato;
la funzione \textit{Tools.getBrowsers()} invoca \textit{Tools.getOs()} e distingue le azioni da intraprendere:

\vspace{0.5cm}
\begin{lstlisting}
public static String getBrowsers(){

	String browsers = "";
	String os = getOs().toLowerCase();
	if(os.contains("linux")){
		[...]
	} else if(os.contains("windows")){
		[...]
	} else if(os.contains("mac")){
		[...]
	} else {
		/**Couldn't recognize OS*/
	}
	return browsers;
\end{lstlisting}

\subsection{Linux}
I browser ritenuti pi\`{u} comuni in ambiente linux sono stati:
\begin{itemize}
	\item Google Chrome
	\item Mozilla Firefox
	\item Opera
	\item Chromium
\end{itemize}
Per identificare l'eventuale versione installata di ogni browser, viene avviato un nuovo processo che esegue la \textit{bash} invocando il programma relativo ad ogni browser con il parametro -\textit{-version}.

\vspace{0.5cm}
\begin{lstlisting}
	[...]
	String tmp = linuxTermOut("firefox --version");
	[...]

private static String linuxTermOut(String cmd){
    String[] args = new String[] {"/bin/bash", "-c", cmd};
    String out = "";
    try {
	    Process proc = new ProcessBuilder(args).start();
	    BufferedReader br = new BufferedReader(
	    new InputStreamReader(proc.getInputStream()));
	    out = br.readLine();
    } catch (IOException ex) {
	    [...]
    }
    return out;
   }
\end{lstlisting}

\subsection{Windows}
I browser ritenuti pi\`{u} comuni in ambiente linux sono stati:
\begin{itemize}
	\item Internet Explorer
	\item Google Chrome
	\item Mozilla Firefox
\end{itemize}
Per individuare in modo sistematico le versioni installate, si \`{e} scelto di interrogare il registro di sistema di Windows.
Ci\`{o} \`{e} stato possibile grazie all'utilizzo di una libreria esterna, la
\textit{Java Native Access}\footnote{https://github.com/java-native-access/jna\#readme}, in cui il package \textit{com.sun.jna.platform.win32.Advapi32Util} offre un'interfaccia semplice e immediata per l'accesso e la manipolazione dei registri di sistema.

\vspace{0.5cm}
\begin{lstlisting}
[...]
String path = "SOFTWARE\\Microsoft\\Internet Explorer";
String vField = getOs().toLowerCase().equals("windows 8")? "svcVersion" : 	 "Version";
String version = Advapi32Util.registryGetStringValue(   
WinReg.HKEY_LOCAL_MACHINE, path, vField);
[...]
\end{lstlisting}

\subsection{Mac OSx}
I browser ritenuti pi\`{u} comuni in ambiente \textit{Mac OSx} sono stati:
\begin{itemize}
	\item Google Chrome
	\item Mozilla Firefox
	\item Opera
	\item Safari
\end{itemize}
Come per \textit{Linux}, per identificare l'eventuale versione installata di ogni browser, viene avviato un nuovo processo che esegue la shell di sistema avviando il programma \textit{system\_profiler} con parametro \textit{SPSoftwareDataType}. L'output offerto dal profiler di sistema contiene al suo interno tutto il software installato sulla macchina, compreso numero di versione e autori. 
Tutte le informazioni vengono salvate all'interno del file \textit{mac\_profile.txt}, da cui successivamente vengono estratte le versioni relative al software cercato.

\vspace{0.5cm}
\begin{lstlisting}
[...]
linuxTermOut("system_profiler SPSoftwareDataType > mac_profile.txt");
[...]
String[] args = new String[] {"/bin/bash", "-c", "grep 
-e \"Google Chrome:\" -e \"Firefox:\" -e \"  Opera:\" -e \"Safari:\" 
-A 2 mac_profiler.txt"};
String str = linuxTermOut(args);
\end{lstlisting}

\section{File di configurazione}
Il file di configurazione permette di impostare i parametri principali delle comunicazioni da effettuare verso l'esterno.
Per ogni contatto \`{e} necessario definire una URL, la periodicit\`{a} di contatto (che pu\`{o} essere fissa o scelta randomicamente in un intervallo pre-impostato), il numero massimo dei contatti da effettuare.
\`{E} inoltre possibile impostare un proxy tramite il quale effettuare le connessioni, uno \textit{user agent} differente da quello di default ed un set di condizione sotto le quali non viene effettuata la connessione alla URL specificata.

Il file di configurazione consiste in un file di testo formattato nel seguente modo:

\vspace{0.5cm}
\begin{lstlisting}
	$proxy_ip:proxy_port /**Opzionale*/
	*URL_1
	minimo_intervallo_di_contatto
	massimo_intervallo_di_contatto
	numero_di_contatti_effettuabili
	condizioni_di_sleep
	user_agent
	*URL_2
	[...]
\end{lstlisting}


\subsection{Immissione e scrittura}
Campi formato GUI, textbox e impicci con popup e tasti push e pull

\subsection{Lettura e caricamento}
Checkbox, popup e lettura file.
Popolamento arraylist di URLDetails

\vspace{0.5cm}
\begin{lstlisting}
Parser parser = new Parser(fileConfPath);
try {
	ArrayList <URLDetails> taskList = parser.readConfigurationFile();
	Tools.schedule(taskList);
} catch (IOException ex) {
	ByobSingleton.myLogger.severe("Parser I/O exception");
}
\end{lstlisting}


\section{Lancio}
Dopo aver selezionato un file di configurazione, cliccando sul tasto \textit{Launch} viene eseguita la parte principale del progetto.
L'ArrayList di \textit{URLDetails} viene passato alla funzione \textit{Tools.schedule(ArrayList<URLDetails> task)} che, per ogni elemento estratto dall'ArrayList, crea un'istanza della classe \textit{ByobTask} (che implementa l'interfaccia \textit{Runnable}) e ne richiede la schedulazione allo \textit{Scheduler Executor Service}.

\vspace{0.5cm}
\begin{lstlisting}
public static void schedule(ArrayList <URLDetails> task) {
    for(int i = 0; i < task.size(); i++) {
	    ByobSingleton.ses.schedule(new ByobTask(task.get(i)), 0,
			    TimeUnit.MILLISECONDS);
    }
}
\end{lstlisting}


\subsection{Schedulazione dei task}
ByobTask rappresenta il singolo task che deve essere eseguito dallo scheduler; ogni task \`{e} legato ad una connessione, ovvero ad un'istanza di \textit{URLDetails}, che contiene tutti i dettagli delle comunicazioni da effettuare.
Nel metodo \textit{run()}, che viene sovrascritto dalla classe, vengono dapprima controllate le condizioni di \textit{sleep}: 
se una di queste risulta verificata il task viene ri-schedulato dopo un intervallo che va da 30 a 45 minuti, al termine dei quali viene effettuato nuovamente il check delle condizioni;
se nessuna condizione \`{e} verificata, allora viene decrementato il numero di contatti ancora da effettuare, viene ri-schedulato il task in esame ed eventualmente viene inviata una GET http alla URL target.
I dettagli del contatto avvenuto sono scritti sul file di log, cos\`{\i} come l'eventuale risposta (qualora specificato) da parte del server.

\vspace{0.5cm}
\begin{lstlisting}
public class ByobTask implements Runnable {

final static ScheduledExecutorService ses = ByobSingleton.getInstance().ses;
URLDetails contact;
[...]

   @Override
   public void run() {
   
	   if(contact.sleepMode()) {
		   /**Sleep mode: try again in 30/45 minutes */
		   int minTimeRestInterval = 30; //Minutes
		   int maxTimeRestInterval = 45; //Minutes
		   long randomInterval = minTimeRestInterval + 
		   random.nextInt(maxTimeRestInterval - minTimeRestInterval + 1);
		   [...]
		   synchronized(ses){
			   ses.schedule(this, randomInterval, TimeUnit.MINUTES);
		   }
	   } else {        
		   /**Synchronized function*/
		   if (contact.decreaseContactNum() < 0) 
			   return; 
		   else if(contact.getContactsNum() > 0){
			   [...]
			   synchronized(ses){
				   ses.schedule(this, (long)randomInterval,
				    TimeUnit.MILLISECONDS);
			   }
		   }
		   /**Write to log file*/
		   [...]
		   int code = ByobComm.httpGet(contact.getURL(),
		    contact.getUserAgent(), URLDetails.proxyIp, 
		    URLDetails.proxyPort, contact.waitForResponse);
	       [...]
	   }
	}
}
\end{lstlisting}

\subsection{Comunicazione HTTP}
I contatti (http GET) sono effettuati tramite chiamata a una funzione statica della classe \textit{ByobComm}.
Il metodo \textit{httpGet([...])} permette di specificare, oltre alla URL da contattare, anche uno user agent personalizzato, l'indirizzo ip e la porta di un server proxy che si desidera utilizzare.

\vspace{0.5cm}
\begin{lstlisting}
static int httpGet(String url, String userAgent, String proxyIp, 
					int proxyPort, Boolean waitForResponse) {  
    String charset = "UTF-8"; 
    HttpURLConnection connection;
    try {
	    if(proxyPort > 0){
		    Proxy proxy = new Proxy(Proxy.Type.HTTP, 
			    new InetSocketAddress(proxyIp, proxyPort));
		    connection = (HttpURLConnection) 
			    new URL(url).openConnection(proxy);
	    } else {
		    connection = (HttpURLConnection) new URL(url).openConnection();
	    }
	    connection.setRequestMethod("GET");
	    connection.setRequestProperty("Accept-Charset", charset);
	    
	    if(!userAgent.isEmpty())
		    connection.setRequestProperty("User-Agent", userAgent);
	    else
		    connection.setRequestProperty("User-Agent", "");
	    connection.connect();
	    int ret = waitForResponse ? connection.getResponseCode() : 0;
	    connection.disconnect();
	    return ret;
	    
    } catch (MalformedURLException ex) {
	    ByobSingleton.getInstance().myLogger.severe("MalformedURLException");
	    return -1;
    
    } catch (IOException ex) {
	    ByobSingleton.getInstance().myLogger.severe("IOException");
	    return -2; 
    }   
}
\end{lstlisting} 




