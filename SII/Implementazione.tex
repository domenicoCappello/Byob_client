\chapter{Implementazione}
\label{chap:implementazione}
\begin{minipage}{12cm}\textit{Sono di seguito presentati dettagli riguardanti l'implementazione dei passi principali eseguiti dal programma in esame. }
\end{minipage}

\section{Avvio iniziale}
All'avvio dell'applicazione, viene visualizzata la schermata principale di configurazione; vengono inoltre raccolte e scritte sul file \textit{sys\_info.txt} informazioni riguardanti il sistema operativo della macchina e le versioni dei browser installati.

\vspace{0.5cm}
\begin{lstlisting}[caption=Funzione main,captionpos=b]
public class Byob_v1 {

	public static void main(String[] args) {
		/**Start the GUI*/
		GUI frame = new GUI();
		final Toolkit toolkit = Toolkit.getDefaultToolkit();
		final Dimension screenSize = toolkit.getScreenSize();
		int x = (screenSize.width - frame.getWidth()) / 2;
		int y = (screenSize.height - frame.getHeight()) / 2;
		frame.setTitle("BYOB v_1");
		frame.setLocation(x, y);
		frame.setVisible(true);
		/**Gather system informations and write them on sys_info.txt*/
		Tools.writeInfoFile("sys_info.txt");
	}

}
\end{lstlisting}

\subsection{Informazioni di sistema}

Il sistema operativo in esecuzione sulla macchina \`{e} restituito dalla funzione \textit{Tools.getOs()}.

\vspace{0.5cm}
\begin{lstlisting}
    public static String getOs(){
	    return System.getProperty("os.name");
    }
\end{lstlisting}

Per riuscire ad identificare i browser installati, sono state adottate strategie differenti per ogni sistema operativo individuato;
la funzione \textit{Tools.getBrowsers()} invoca \textit{Tools.getOs()} e distingue le azioni da intraprendere:

\vspace{0.5cm}
\begin{lstlisting}
public static String getBrowsers(){

	String browsers = "";
	String os = getOs().toLowerCase();
	if(os.contains("linux")){
		...
	} else if(os.contains("windows")){
		...
	} else if(os.contains("mac")){
		...
	} else {
		/**Couldn't recognize OS*/
	}
	return browsers;
\end{lstlisting}

\subsubsection{Linux}
I browser ritenuti pi\`{u} comuni in ambiente linux sono stati:
\begin{itemize}
	\item Google Chrome
	\item Mozilla Firefox
	\item Opera
	\item Chromium
\end{itemize}
Per identificare l'eventuale versione installata di ogni browser, viene avviato un altro processo che esegue la \textit{bash} ed invoca ogni 

\subsubsection{Windows}
Scelta browser, lettura file di registro mediante libreria esterna, listato.
Impicci Chrome

\subsubsection{Mac osx}
Invocazione bash, scelta browser, uso del system\_profiler e file mac\_profile.txt

\section{File di configurazione}
File un botto fico, fa impostare tante belle cose (url, intervallo, etc.) Definito da -> Listato: proxy, url etc.

\subsection{Immissione e scrittura}
Campi formato GUI, textbox e impicci con popup e tasti push e pull

\subsection{Lettura e caricamento}
Checkbox, popup e lettura file.
Popolamento arraylist di URLDetails

\section{Lancio}
Si parte.

\subsection{Schedulazione dei task}
Listato ByobTask

\subsection{Comunicazione HTTP}
listato ByobComm, waitForResponse opzionale, sincronizzazione.



