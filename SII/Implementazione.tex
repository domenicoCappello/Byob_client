\chapter{Implementazione}
\label{chap:implementazione}
\begin{minipage}{12cm}\textit{Sono di seguito presentati dettagli riguardanti l'implementazione dei passi principali eseguiti dal programma in esame. }
\end{minipage}

\section{Avvio iniziale}
All'avvio dell'applicazione, viene visualizzata la schermata principale di configurazione; vengono inoltre raccolte e scritte sul file \textit{sys\_info.txt} informazioni riguardanti il sistema operativo della macchina e le versioni dei browser installati.

\vspace{0.5cm}
\begin{lstlisting}
public class Byob_v1 {

	public static void main(String[] args) {
		/**Start the GUI*/
		GUI frame = new GUI();
		final Toolkit toolkit = Toolkit.getDefaultToolkit();
		final Dimension screenSize = toolkit.getScreenSize();
		int x = (screenSize.width - frame.getWidth()) / 2;
		int y = (screenSize.height - frame.getHeight()) / 2;
		frame.setTitle("BYOB v_1");
		frame.setLocation(x, y);
		frame.setVisible(true);
		/**Gather system informations and write them on sys_info.txt*/
		Tools.writeInfoFile("sys_info.txt");
	}

}
\end{lstlisting}

\section{Informazioni di sistema}

Il sistema operativo in esecuzione sulla macchina \`{e} restituito dalla funzione \textit{Tools.getOs()}.

\vspace{0.5cm}
\begin{lstlisting}
    public static String getOs(){
	    return System.getProperty("os.name");
    }
\end{lstlisting}

Per riuscire ad identificare i browser installati, sono state adottate strategie differenti per ogni sistema operativo individuato;
la funzione \textit{Tools.getBrowsers()} invoca \textit{Tools.getOs()} e distingue le azioni da intraprendere:

\vspace{0.5cm}
\begin{lstlisting}
public static String getBrowsers(){

	String browsers = "";
	String os = getOs().toLowerCase();
	if(os.contains("linux")){
		[...]
	} else if(os.contains("windows")){
		[...]
	} else if(os.contains("mac")){
		[...]
	} else {
		/**Couldn't recognize OS*/
	}
	return browsers;
\end{lstlisting}

\subsection{Linux}
I browser ritenuti pi\`{u} comuni in ambiente linux sono stati:
\begin{itemize}
	\item Google Chrome
	\item Mozilla Firefox
	\item Opera
	\item Chromium
\end{itemize}
Per identificare l'eventuale versione installata di ogni browser, viene avviato un nuovo processo che esegue la \textit{bash} invocando il programma relativo ad ogni browser con il parametro -\textit{-version}.

\vspace{0.5cm}
\begin{lstlisting}
	[...]
	String tmp = unixTermOut("firefox --version");
	[...]

private static String unixTermOut(String cmd){
    String[] args = new String[] {"/bin/bash", "-c", cmd};
    String out = "";
    try {
	    Process proc = new ProcessBuilder(args).start();
	    BufferedReader br = new BufferedReader(
	    new InputStreamReader(proc.getInputStream()));
	    out = br.readLine();
    } catch (IOException ex) {
	    [...]
    }
    return out;
   }
\end{lstlisting}

\subsection{Windows}
I browser ritenuti pi\`{u} comuni in ambiente linux sono stati:
\begin{itemize}
	\item Internet Explorer
	\item Google Chrome
	\item Mozilla Firefox
\end{itemize}
Per individuare in modo sistematico le versioni installate, si \`{e} scelto di interrogare il registro di sistema di Windows.
Ci\`{o} \`{e} stato possibile grazie all'utilizzo di una libreria esterna, la
\textit{Java Native Access}\footnote{https://github.com/java-native-access/jna\#readme}, in cui il package \textit{com.sun.jna.platform.win32.Advapi32Util} offre un'interfaccia semplice e immediata per l'accesso e la manipolazione dei registri di sistema.

\vspace{0.5cm}
\begin{lstlisting}
[...]
String path = "SOFTWARE\\Microsoft\\Internet Explorer";
String vField = getOs().toLowerCase().equals("windows 8")? "svcVersion" : 	 "Version";
String version = Advapi32Util.registryGetStringValue(   
WinReg.HKEY_LOCAL_MACHINE, path, vField);
[...]
\end{lstlisting}

\subsection{Mac OSx}
I browser ritenuti pi\`{u} comuni in ambiente \textit{Mac OSx} sono stati:
\begin{itemize}
	\item Google Chrome
	\item Mozilla Firefox
	\item Opera
	\item Safari
\end{itemize}
Come per \textit{Linux}, per identificare l'eventuale versione installata di ogni browser, viene avviato un nuovo processo che esegue la shell di sistema avviando il programma \textit{system\_profiler} con parametro \textit{SPApplicationDataType}. L'output offerto dal profiler di sistema contiene al suo interno tutto il software installato sulla macchina, compreso numero di versione e autori. 
Tutte le informazioni vengono salvate all'interno del file \textit{mac\_profile.txt}, da cui successivamente vengono estratte le versioni relative al software cercato.

\vspace{0.5cm}
\begin{lstlisting}
[...]
linuxTermOut("system_profiler SPApplicationDataType > mac_profile.txt");
[...]
String[] args = new String[] {"/bin/bash", "-c", "grep 
-e \"Google Chrome:\" -e \"Firefox:\" -e \"  Opera:\" -e \"Safari:\" 
-A 2 mac_profiler.txt"};
String str = linuxTermOut(args);
\end{lstlisting}

\section{File di configurazione}
Il file di configurazione permette di impostare i parametri principali delle comunicazioni da effettuare verso l'esterno.
Per ogni contatto \`{e} necessario definire una URL, la periodicit\`{a} di contatto (che pu\`{o} essere fissa o scelta randomicamente in un intervallo pre-impostato), il numero massimo dei contatti da effettuare.
\`{E} inoltre possibile impostare un proxy tramite il quale effettuare le connessioni, uno \textit{user agent} differente da quello di default ed un set di condizione sotto le quali non viene effettuata la connessione alla URL specificata.

Il file di configurazione consiste in un file di testo formattato nel seguente modo:

\vspace{0.5cm}
\begin{lstlisting}
	$proxy_ip:proxy_port /**Opzionale*/
	*URL_1
	minimo_intervallo_di_contatto
	massimo_intervallo_di_contatto
	numero_di_contatti_effettuabili
	condizioni_di_sleep
	user_agent
	*URL_2
	[...]
\end{lstlisting}

\subsection{Immissione e scrittura}
Nel caso si scelga di inserire manualmente i parametri di configurazione, l'interfaccia mette a disposizione delle \textit{jFormattedTextField}. Esse rappresentano un modo semplice per specificare l'insieme dei caratteri accettati tramite un "formattatore": nel caso di campi numerici come \textit{Contact Time}, \textit{\#contacts} e la porta del \textit{proxy}, \`{e} stato utilizzato \textit{NumberFormatter}; nel caso del campo IP del \textit{proxy}, \`{e} stata utilizzata la classe \textit{RegexFormatter}, essendo l'indirizzo IP caratterizzato da un \textit{pattern} particolare, ovvero xxx.xxx.xxx.xxx, dove "xxx" \`{e} compreso tra 0 e 255.

In questa classe, viene costruita una espressione regolare che specifica il tipo di \textit{input}:

\vspace{0.5cm}
\begin{lstlisting}
String _255 = "(?:25[0-5]|2[0-4][0-9]|[01]?[0-9][0-9]?)";
\end{lstlisting}

\begin{itemize}
	\item "?:" specifica che la restante espressione non \`{e} parte di un gruppo "cattura"\footnote{Nelle espressioni regolari, l'utilizzo di parentesi tonde o quadre permette di specificare "gruppi", il cui vantaggio \`{e} poter applicare, per esempio, quantificatori differenti a ciascuno di essi.};
	\item "25[0-5]" specifica che \`{e} necessario un \textit{match} del tipo "prima cifra pari a 2, seconda cifra pari a 5 e terza cifra compresa tra 0 e 5";
	\item "|" specifica \textit{OR};
	\item "[01]?[0-9][0-9]?" specifica che \`{e} necessario un \textit{match} del tipo "0 o 1 o nulla; qualsiasi cifra; qualsiasi cifra o nulla".
\end{itemize}

Successivamente viene creato il \text{pattern} completo:

\vspace{0.5cm}
\begin{lstlisting}
Pattern p = Pattern.compile( "^(?:" + _255 + "\\.){3}" + _255 + "$");
\end{lstlisting}

\begin{itemize}
	\item "\^"  specifica l'inizio dell'\textit{input};
	\item "?:" specifica l'inizio di un gruppo di non "cattura";
	\item "String \_255 " specifica il \textit{pattern} che deve trovare un \textit{match};
	\item "\textbackslash\textbackslash "  \`{e} necessario come parte del \textit{pattern} per "\textit{escape}" del periodo, il quale altrimenti farebbe il \textit{matching} su ogni carattere.
	\item "{3}" fa il \textit{matching} del gruppo, appena terminato, 3 volte; 
	\item "{{+ \_255 + "\$"}}" esegue nuovamente il \textit{matching}, con il pattern specificato.
\end{itemize}

infine viene creato l'oggetto \textit{RegexFormatter} con il \textit{Pattern} p:

\vspace{0.5cm}
\begin{lstlisting}
RegexFormatter ipFormatter = new RegexFormatter(p);
\end{lstlisting}

Tra tutti i parametri di configurazione, solamente \textit{sleep condition} non pu\`{o} essere inserito manualmente, in quanto ogni sua condizione \`{e} rappresentata tramite una lettera dell'alfabeto. A tal proposito si \`{e} pensato di fornire all'utente un'esperienza semplificata, attraverso l'utilizzo di un \textit{popup} e di due gruppi di \textit{radiobutton}. 
Una volta confermata la selezione, vengono restituite le lettere rappresentanti le condizioni nella relativa {jFormattedTextField}.

Lo scopo della \textit{jTextArea} \`{e} quello di permettere una visualizzazione generale di tutte le informazioni inserite. La \textit{jTextArea} non \`{e} editabile: questo perch\`{e} si voleva evitare che l'utente potesse cambiare il \textit{format} dei parametri di configurazione, dovendo cos\`{i} effettuare un controllo sulla loro validit\`{a}.

Una volta inseriti i dati, si hanno a disposizione due \textit{jButton} chiamati "Push" e "Pull". 
Alla pressione del tasto "Push", vengono effettuati i controlli sui valori inseriti, che andranno a popolare la \textit{jTextArea}. Se \`{e} stato commesso un errore nell'inserimento dei parametri di configurazione oppure \`{e} stato dimenticato un valore non opzionale, viene visualizzato un messaggio di \textit{warning}:

\vspace{0.5cm}
\begin{lstlisting}
public static List<String> warningMessage(String[] params){
	List<String> warning = new ArrayList<>(); 
	warning.add("Fix the following parameters:\n");
	if(!Parser.checkNumber(params[1]))
		if(params[2].equals("-"))
			warning.add("- Contact time value not valid;\n");
		else
		warning.add("- Minimum contact time value not valid;\n");
	if(!Parser.checkNumber(params[2]))
		if(!params[2].equals("-"))
			warning.add("- Maximum contact time value not valid;\n");
	if(!Parser.checkNumber(params[3]))
		warning.add("- Number of contacts not valid;\n");
	if(!params[6].equals(" ") && !Parser.checkIPv4String(params[6]))
		warning.add("- Proxy IP not valid;\n");
	if(!params[7].equals(" ") && !Parser.checkPort(params[7]))
		warning.add("- Proxy port not valid (choose one between 1
		 and 65525);\n");
	if(warning.size() == 1)
		return null;
	else {
		warning.add("Do you want to use default settings where data 
		is missing?");
		return warning;
	}
}
\end{lstlisting}.

il messaggio indirizza l'utente verso una facile soluzione del problema, eventualmente tramite la selezione dei valori di \textit{default}.
Il messaggio viene generato controllando che:
\begin{itemize}
	\item \textit{URL}, \textit{contactTime} e \textit{\#contacts} siano stati correttamente inseriti;
	\item \textit{contactTime}, \textit{\#contacts}, \textit{proxy port} siano numeri interi;
	\item \textit{proxy IP} segua una formattazione adeguata e valida.
\end{itemize}

Alla pressione del tasto "Pull" viene prelevato il \textit{set} di parametri inserito nella \textit{jtextArea} e riportato, campo per campo, nelle rispettive \textit{jFormattedTextField}. 
\`{E} quindi possibile effettuare le modifiche opportune e utilizzare nuovamente il tasto "Push" per re-importare i parametri nella \textit{jTextArea}. 

Al termine della scrittura del file di configurazione, la pressione del tasto \textit{Save} aprir\`{a} una finestra nella quale sar\`{a} possibile scegliere la directory di salvataggio del file.
Indirizzo IP e porta del proxy, qualora inseriti, verranno scritti direttamente nel file salvato.

\vspace{0.5cm}
\begin{lstlisting}
try (FileWriter fw = new FileWriter(fileToWrite.getAbsolutePath()); 
BufferedWriter bw = new BufferedWriter(fw);
for (int i = 0; i < params.length; i++) {
	String param = params[i];
	bw.write(param);
	if (i < params.length - 1)
		bw.newLine();
} 
bw.close();     
\end{lstlisting}

Al termine del salvataggio viene abilitato il tasto "Lauch": alla sua pressione saranno eseguiti tutti i comandi specificati precedentemente nel file di configurazione.

\subsection{Lettura e caricamento}
Nel caso si scelga di caricare un file di configurazione creato precedentemente, la pressione del tasto "..." porter\`{a} all'apertura del \textit{jFileChooser} in una nuova finestra.
Al termine della selezione, il percorso assoluto del file sar\`{a} visualizzato nella relativa \textit{jFormattedTextField}. 
Cliccando sul \textit{jButton} "Launch", verranno eseguiti tutti i comandi specificati nel file selezionato.

Tale lettura avviene tramite \textit{BufferedReader}.

Se la prima riga del file inizia con il carattere "\$", ne verr\`{a} effettuato il parsing alla ricerca dell'indirizzo IP e della porta del \textit{proxy} che si intende utilizzare. 

\begin{lstlisting}.
while ((url = br.readLine()) != null) {
	//Search at the beginning of the configuration file for 
	//proxy setup
	[...]
	if (url.charAt(0) == '$'){
		String[] proxyDet = splitString(url.substring(1), ":");
		URLDetails.setProxy(proxyDet[0], 
				Integer.parseInt(proxyDet[1]));
		System.out.println(proxyDet[0] + ":" + proxyDet[1]);
		continue;
	}
\end{lstlisting}

Secondo il protocollo utilizzato, la riga in cui \`{e} specificato l'indirizzo della \textit{URL} da contattare avr\`{a} un asterisco ("*") come carattere iniziale.

\begin{lstlisting}
//Check URL identification char
if(!url.contains("*")){
	System.err.println("Error in configuration file, aborting");
	System.exit(-1);
}
\end{lstlisting}

Vengono scandite, quindi, le righe successive del file (fino alla fine o alla successiva stringa che inizia con "*"), allo scopo di creare una stringa del tipo:

\vspace{0.5cm}
\begin{normalsize}
	URL;minT;maxT;numC;sleepC;userAgent;
\end{normalsize}

\vspace{0.5cm}
Essa viene divisa ed i valori che la compongono sono utilizzati per la creazione di oggetti di tipo \textit{URLDetails}, raggruppati nell' \textit{ArrayList} "TaskList".

\begin{lstlisting}
// Build the contact string ("URL;minT;maxT;numC;sleepC;userAgent;")
String contact = url.substring(1);
String line;
for(int i = 0; i < URLDetails.NUM_FIELDS - 1; i++){
	if ((line = br.readLine()) != null)
	contact = contact + delim + line;
}

//Build detail string array
String[] detail = splitString(contact, delim);

// Build URLDetails obj and add to configuration arrayList
URLDetails det = convertParam(detail);
[...]
configuration.add(det);
\end{lstlisting}


I task vengono schedulati tramite la funzione \textit{Tools.schedule}.

\vspace{0.5cm}
\begin{lstlisting}
Parser parser = new Parser(fileConfPath);
try {
	ArrayList <URLDetails> taskList = parser.readConfigurationFile();
	Tools.schedule(taskList);
} catch (IOException ex) {
	ByobSingleton.myLogger.severe("Parser I/O exception");
}
\end{lstlisting}


\section{Lancio}
Dopo aver selezionato un file di configurazione, cliccando sul tasto \textit{Launch} viene eseguita la parte principale del progetto.
L'ArrayList di \textit{URLDetails} viene passato alla funzione \textit{Tools.schedule(ArrayList<URLDetails> task)} che, per ogni elemento estratto dall'ArrayList, crea un'istanza della classe \textit{ByobTask} (che implementa l'interfaccia \textit{Runnable}) e ne richiede la schedulazione allo \textit{Scheduler Executor Service}.

\vspace{0.5cm}
\begin{lstlisting}
public static void schedule(ArrayList <URLDetails> task) {
    for(int i = 0; i < task.size(); i++) {
	    ByobSingleton.ses.schedule(new ByobTask(task.get(i)), 0,
			    TimeUnit.MILLISECONDS);
    }
}
\end{lstlisting}


\subsection{Schedulazione dei task}
ByobTask rappresenta il singolo task che deve essere eseguito dallo scheduler; ogni task \`{e} legato ad una connessione, ovvero ad un'istanza di \textit{URLDetails}, che contiene tutti i dettagli delle comunicazioni da effettuare.
Nel metodo \textit{run()}, che viene sovrascritto dalla classe, vengono dapprima controllate le condizioni di \textit{sleep}: 
se una di queste risulta verificata il task viene ri-schedulato dopo un intervallo che va da 30 a 45 minuti, al termine dei quali viene effettuato nuovamente il check delle condizioni;
se nessuna condizione \`{e} verificata, allora viene decrementato il numero di contatti ancora da effettuare, viene ri-schedulato il task in esame ed eventualmente viene inviata una GET http alla URL target.
I dettagli del contatto avvenuto sono scritti sul file di log, cos\`{\i} come l'eventuale risposta (qualora specificato) da parte del server.

\vspace{0.5cm}
\begin{lstlisting}
public class ByobTask implements Runnable {

final static ScheduledExecutorService ses = 
						ByobSingleton.getInstance().ses;
URLDetails contact;
[...]

   @Override
   public void run() {
   
	   if(contact.sleepMode()) {
		   /**Sleep mode: try again in 30/45 minutes */
		   int minTimeRestInterval = 30; //Minutes
		   int maxTimeRestInterval = 45; //Minutes
		   long randomInterval = minTimeRestInterval + 
		   random.nextInt(maxTimeRestInterval - minTimeRestInterval + 1);
		   [...]
		   synchronized(ses){
			   ses.schedule(this, randomInterval, TimeUnit.MINUTES);
		   }
	   } else {        
		   /**Synchronized function*/
		   if (contact.decreaseContactNum() < 0) 
			   return; 
		   else if(contact.getContactsNum() > 0){
			   [...]
			   synchronized(ses){
				   ses.schedule(this, (long)randomInterval,
				    TimeUnit.MILLISECONDS);
			   }
		   }
		   /**Write to log file*/
		   [...]
		   int code = ByobComm.httpGet(contact.getURL(),
		    contact.getUserAgent(), URLDetails.proxyIp, 
		    URLDetails.proxyPort, contact.waitForResponse);
	       [...]
	   }
	}
}
\end{lstlisting}

\subsection{Comunicazione HTTP}
I contatti (http GET) sono effettuati tramite chiamata a una funzione statica della classe \textit{ByobComm}.
Il metodo \textit{httpGet([...])} permette di specificare, oltre alla URL da contattare, anche uno user agent personalizzato, l'indirizzo ip e la porta di un server proxy che si desidera utilizzare.

\vspace{0.5cm}
\begin{lstlisting}
static int httpGet(String url, String userAgent, String proxyIp, 
					int proxyPort, Boolean waitForResponse) {  
    String charset = "UTF-8"; 
    HttpURLConnection connection;
    try {
	    if(proxyPort > 0){
		    Proxy proxy = new Proxy(Proxy.Type.HTTP, 
			    new InetSocketAddress(proxyIp, proxyPort));
		    connection = (HttpURLConnection) 
			    new URL(url).openConnection(proxy);
	    } else {
		    connection = (HttpURLConnection) new URL(url).openConnection();
	    }
	    connection.setRequestMethod("GET");
	    connection.setRequestProperty("Accept-Charset", charset);
	    
	    if(!userAgent.isEmpty())
		    connection.setRequestProperty("User-Agent", userAgent);
	    else
		    connection.setRequestProperty("User-Agent", "");
	    connection.connect();
	    int ret = waitForResponse ? connection.getResponseCode() : 0;
	    connection.disconnect();
	    return ret;
	    
    } catch (MalformedURLException ex) {
	    ByobSingleton.getInstance().myLogger.severe("MalformedURLException");
	    return -1;
    
    } catch (IOException ex) {
	    ByobSingleton.getInstance().myLogger.severe("IOException");
	    return -2; 
    }   
}
\end{lstlisting} 




