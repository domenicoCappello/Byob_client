\chapter{Introduzione}
\section{Premessa}
Oggigiorno, la maggior parte dei PC utilizza sistemi operativi senza patch e/o senza alcuna sicurezza dietro un \textit{firewall}, rendendoli facili prede di attacchi diretti, orchestrati da malintenzionati, e di attacchi di tipo indiretto, mascherati dietro programmi che l'utente utilizza costantemente (vedi reti \textit{P2P}).\\
Con l'incremento delle connessioni a banda larga si ha avuto anche un incremento del numero di potenziali vittime di attacchi, con cui i malintenzionati traggono beneficio dalla situazione, utilizzandola a loro vantaggio, sfruttando anche l'automatizzazione di tecniche per la scansione di porzioni della rete che semplifica la ricerca di sistemi vulnerabili. Una volta che un vasto numero di macchine sono state infettate, esse entrano a far parte di una ``rete di macchine compromesse che posso essere controllate da remoto\footnote{Provos Niels, Holz Thorsten (2007).}'', chiamata \textbf{botnet}.\\
Una \textit{botnet} consiste di tre elementi principali che sono i bot (cio\`e le macchine infettate che ne fanno parte), il \textit{command and control server} (C\&C - da cui ogni bot riceve istruzioni e con cui il malintenzionato ha privilegi amministrativi remoti su tutte le macchine infette) ed il \textit{botmaster}; si basa, inoltre, su quattro concetti chiave:
\begin{enumerate}
\item le \textit{botnet} sono reti, quindi sistemi in cui la comunicazione \`e importante;
\item le macchine che fanno parte di una \textit{botnet} sono, tipicamente, partecipanti ignari;
\item i \textit{bot} sono controllabili da remoto, permettendo di fare rapporto o ricevere ordini da una struttura C\&C (centralizzata o decentralizzata);
\item i \textit{bot} sono controllati da persone con intenti malevoli che fanno capo a qualche forma di attivit\`{a} illegale.
\end{enumerate}

\vspace*{1cm}
\section{HTTP Botnet}
\begin{figure}[h]
        \centering
		\includegraphics[width=0.5\linewidth]{./imgs/botnet1999}
        \caption{Struttura di una botnet}
        \label{strutturabotnet}
\end{figure}
Quando il protocollo HTTP nacque nel 1999, nessuno avrebbe mai pensato che sarebbe stato utilizzato per le botnet. 
La prima generazione di botnet utilizzava l'\textit{Internet Relay Chat} (IRC\footnote{Protocollo di messaggistica istantanea su Internet, che consente sia la comunicazione diretta fra due utenti, che il dialogo contemporaneo di gruppi di persone raggruppati in stanze di discussione dette canali.}) e relativi canali per instaurare un meccanismo di ``controllo e comando''. I bot IRC seguono lo stesso approccio PUSH di quando ci si unisce ai canali, rimanendo connessi. Essi si connettono ai server IRC e ai canali che sono stati selezionati dal \textit{botmaster} e attendono comandi. 
Invece di rimanere connessi, i bot HTTP controllano periodicamente per aggiornamenti oppure nuovi comandi: questo modello \`e detto di PULL e continua ad intervalli regolari definiti dal botmaster, che usa il protocollo HTTP per nascondere le proprie attivit\`{a} tra il normale flusso web, riuscendo ad evitare facilmente i metodi di rivelazione come i \textit{firewall}. 

\vspace*{1cm}
\section{Obiettivo}
L'obiettivo \`e quello di sviluppare un software per workstation che definisca bot in grado di contattare delle URL a cui sono associati una serie di parametri fondamentali:
\begin{enumerate}
\item la periodicit\`{a} di contatto (fissa o variabile in un intervallo temporale);
\item il numero massimo di contatti da effettuare;
\item una eventuale modalit\`{a} di ``\textit{sleep}'', intesa come insieme di condizioni in cui non viene effettuata nessuna azione ;
\item un eventuale user-agent ``\textit{custom}'';
\item l'indirizzo ip e la porta di un eventuale ``\textit{proxy}'' pubblico.
\end{enumerate}
Tali parametri saranno impostati tramite un file di testo (pre-compilato oppure configurabile mediante una \textit{Graphic User Interface}).\\
I contatti effettuati e i parametri di configurazione in uso verranno salvati su un file di log;
le informazioni principali relative alla macchina su cui il codice \`{e} in esecuzione saranno salvate nel file \textit{sys\_info.txt}.

%\vspace*{1cm}
%\section{Struttura della relazione}
%L'analisi, la progettazione e  sviluppo del progetto inizia con un'esposizione del programma implementato, spiegandone la logica, ed esponendo il tentativo di conferire una modalit\'a di utilizzo immediata, con poche piccole azioni. Nel capitolo successivo si parler\'a degli elementi principali del progetto e tutto ci\`o che li riguarda, evidenziandone il processo di analisi. A seguire, un capitolo dedicato alle metodologie utilizzate per la creazione dei bot facenti parti della rete. Infine, l'ultimo punto, \`e quello dell'analisi dei casi di test effettuati per un corretto funzionamento del programma.